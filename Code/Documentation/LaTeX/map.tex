\section{\texttt{Map} Class}
The user can simulate an environment by creating a \texttt{Map} class instance, which contains object arrays of \texttt{Point}, \texttt{Entity}, \texttt{Object} and \texttt{Constraint} class instances.
Each point, entity and object will eventually form a single vertex in the graph (unless they are unobserved and removed). Each constraint will eventually form a single edge in the graph (unless it is unobserved and removed).

\subsection{\texttt{Point} Class}
A single point can be initialised with the \texttt{Point} class constructor, requiring either no inputs for preallocation, or a trajectory and index.
Groups of points are initialised with a \texttt{Map} class method. For a set of $n$ points, and a simulation with $m$ time steps, \texttt{pointPositions} should be an array of size $3n \times m$, where each $3 \times m$ block corresponds to the $[x,y,z]^T$ trajectory of a single point
\begin{lstlisting}
map = map.initialisePoints(pointPositions);
\end{lstlisting}

\subsection{\texttt{Entity} Class}
Entities are initialised similarly to points, but with an additional parameter \texttt{type}.
\begin{lstlisting}
map = map.initialiseEntities(entityTypes,entityParameters);
\end{lstlisting}

\subsection{\texttt{Object} Class}
Objects are initialised similarly to entities, but with an additional input \texttt{pose}.
\begin{lstlisting}
map = map.initialiseObjects(objectPoses,objectTypes,...
							objectParameters);
\end{lstlisting}

\subsection{\texttt{Constraint} Class}
Constraints are initialised by passing a cell array to the class constructor.
Each row of the $n \times 6$ cell array represents a single constraint.
map = map.initialiseConstraints(constraints);

Each row gives:
$\{i_{objects},i_{parentEntities},i_{childEntities},i_{points},type,value\}$\\
The first four entries correspond to indexes of features involved in the constraint.
For example, a point-plane constraint between point 1 and plane 1 could be initialised with\\
$\{[],1,[],1,\texttt{'point-plane'},0\}$ \\
where the value of $0$ corresponds to a distance of $0$ between the point and the plane.

A constraint enforcing planes $1$ and $2$ as parallel would be initialised with:\\
$\{[],[],[1,2],[],\texttt{'plane-plane-fixedAngle'},1\}$\\
where the value corresponds to a dot product of $1$ for the plane normals.

A loose constraint of this angle, also estimating the angle itself as an entity with index $3$ would be initialised with:\\
$\{[],3[],[1,2],[],\texttt{'plane-plane-angle'},1\}$
